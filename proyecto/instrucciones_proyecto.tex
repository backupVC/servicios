\documentclass{article}
\usepackage[utf8]{inputenc}
\usepackage[spanish]{babel}

\title{PROYECTO BACKEND}
\author{Programación de Procesos y Servicios  \\
	Departamento de Informática \\
	IES Virgen del Carmen
	}

\date{\today}
% Hint: \title{what ever}, \author{who care} and \date{when ever} could stand 
% before or after the \begin{document} command 
% BUT the \maketitle command MUST come AFTER the \begin{document} command! 
\begin{document}

\maketitle


\begin{abstract}
Realización de un backend usando \emph{node.js}, \emph{express}, \emph{mongoose} y \emph{jwt}
\end{abstract}

\section{Introducción}
El proyecto corresponde al primero a realizar en este módulo donde se valorará las destrezas desarrolladas durante este primer trimeste.

\paragraph{Tecnologías}
Las tecnologías a emplear son:
\begin{itemize}
\item javascritp ES6
\item node.js
\item express
\item mongoose
\item mongodb
\item docker
\end{itemize}

\newpage

\section{Requisitos funcionales}

\begin{itemize}
\item El proyecto se realizará de forma individual.
\item Podrás servir de backend para otros proyectos de otro módulos.
\item El lenguaje de programación a emplear es \emph{javascript}
\item Se realizará en el estándar \emph{ECMAScript 6}
\item Se utilizará \emph{REST} como arquitectura de la aplicación.
\item Se usará para la persistencia una base de datos \emph{NoSQL}, mas concretamente \emph{MongoDB}
\item Se utilizará el framework \emph{express} para el desarrollo del servicio web.
\item Se usará la biblioteca \emph{jwt} para \emph{node} para llevar acabo la autenticación.
\item Se utilizará un patrón similar a \emph{MVC}
\item Se definirán dos tipos de rutas, una para la autenticación y otra para la aplicación en particular
\item Las rutas para la autenticación del usuario constarán, una para alta del usuario y otra para realizar el \emph{login}
\item El \emph{login} ser realizará mediante un email y el nombre del usuario
\item La rutas para la aplicación, deberán abarcar todas las operaciones \emph{CRUD} 
\item Se debe modularizar al máximo la aplicación, por lo que las rutas, controladores y modelos deben estar los mas desacoplados posibles por lo que deben realizarse en carpetas distintas.
\item La configuración de acceso a  la base de datos se hará en un fichero dentro de una carpeta \emph{config}.
\item Todos los \emph{endpoints} irán en inglés y relacionado con el significado que realiza cada acción.
\item Para el modelo se usará \emph{mongoose} para realizar el esquema de acceso a la base de datos. En dicho esquema deberá incluir restricciones como (aquellas que sean necesarias):
\newpage
\begin{enumerate}
\item Campos requeridos.
\item Valores por defecto.
\item Expresiones regulares.
\item \emph{enum} que incluya un conjunto de valores.
\item Restricciones en valores numéricos
\item \dots 
\end{enumerate}
\item Se usará un repositorio de \emph{git} para el desarrollo del proyecto. Los \emph{commits} deben ser claros y relacionado con lo que se está haciendo. Se llevarán acabo cuando se termine una acción muy determina, por ejemplo finalización del modelo, actualización del controlador con una nueva función, \dots
\item Se usará una imagen \emph{docker} que englobe la aplicación usando un archivo \emph{Dockerfile}
\item Se subirá dicha imagen a \emph{Docker Hub}
\item Se creará un \emph{microservicio} que englobe la \emph{app} y \emph{mongo}, para ello usaremos 
\end{itemize}





\section{Entrega}
Se entregará los siguientes documentos:
\begin{itemize}
\item Un documento de texto que incluya una breve descripción del proyecto, incluyendo una tabla que explique la arquitectura REST a emplear, asi como una descripción del modelo. También se incluirá la URL del gitHub o cualquier otro repositorio donde se encuentre el proyecto. En fichero \emph{README.md} se incluirá los datos anteriores.
\item En un archivo comprimido se incluirá el proyecto, además un fichero con un array json con valores usados en la BD de mongo usados durante el desarrollo del proyecto.
\item La imagen realizada con DockerFile
\item El despliegue del microservicio.
\end{itemize}

\newpage

\section{Criterios de valoración}
Los criterios para elaborar la calificación del proyecto son los indicados a continuación:
\begin{description}
\item[\emph{1 pto}] El proyecto funciona.
\item[\emph{1 pto}] Si el proyecto está modularizado usando un patrón similar al \emph{MVC}
\item[\emph{1 pto}] Por la complejidad del modelo, donde se tendrá en cuenta restricciones, tipos de datos, \dots
\item[\emph{1 pto}] Uso de un repositorio de \emph{git}, teniendo en cuenta \emph{commits} claros y el fichero \emph{README.md} descriptivo.
\item[\emph{1 pto}] Uso de \emph{E6} en todo el proyecto.
\item[\emph{1 pto}] Si el proyecto utiliza autenticación mediante \emph{token}
\item[\emph{1 pto}] Uso de código limpio, valorándose el uso del inglés en el desarrollo de la aplicación.
\item[\emph{1 pto}] Documentación clara y precisa.
\item[\emph{1 pto}] Por el uso de \emph{Docker}
\item[\emph{1 pto}] Exposición del proyecto, donde quede claro el manejo y conocimiento de las tecnologías empleadas.

\end{description}

\end{document}
