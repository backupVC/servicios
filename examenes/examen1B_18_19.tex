\documentclass[addpoints,12pt]{exam}
\usepackage[spanish]{babel}
%\usepackage[latin1]{inputenc}
\usepackage[utf8x]{inputenc}
\usepackage{hyperref}
\pagestyle{empty}
\begin{document}
\begin{center}
\fbox{\fbox{\parbox{5.5in}{\centering {\LARGE EXAMEN TIPO B}\\Sigue las intrucciones que marca cada ejercicio, no solo para la realización del mismo sino también para la entrega de los mismos.\\\emph{No se permite ningun documento de ayuda, excepto los aportados por el profesor}\\\textbf{No se corregira ejercicios con errores de programación}}}}
\end{center}
\vspace{0.1in}
%\makebox[\textwidth]{Nombre:\enspace\hrulefill}
\begin{questions}
\question(3.5 ptos) Crea una clase en \emph{javascript} que use como argumento un \emph{array} de objetos persona con los siguientes campos (iguales que los datos del fichero de \emph{arrays} de \emph{datos.csv} que se te proporciona):
\begin{itemize}
\item name
\item age
\item email
\item rol
\end{itemize}
Y proporciones los siguientes métodos:
\begin{itemize}
\item Nos de el número de personas con una edad superior a una dada.
\item Nos de un array de objetos persona que tenga una edad y un rol determinado.
\item Nos de un array de email dado un rol determinado.
\end{itemize}
Se aporta un fichero denominado \emph{datos.csv} y un programa \emph{leerCSV.js} que se encarga de leer los datos y crear un array de objeto persona que debes usar para comprobar los métodos de la clase. Este fichero (\emph{leerCSV.js}) no debes modificar la lectura del fichero CSV, el se encarga de eliminar la cabecera, parsear todas las líneas y construye el array de objetos que vas a usar\\
Como datos de comprobación puedes usar:\\
En el caso del primer método, el número de personas mayor que 40 son 55\\
En el segundo caso con 50 años y rol Engineer hay solo dos objetos\\
En el último método si introducimos \emph{Engineer} debes obtener un array de tamaño 53, en el caso de \emph{Webmaster} es de 24

\newpage

\question(3 ptos) Se te proporciona un fichero denominado \emph{datos.json} que es un array de json. Importa dichos datos a la BD de mongo, usa como nombre de BD \emph{test} y coleccion \emph{users}.
\begin{itemize}
\item Obtén todos los resultados de objetos que sean mayores de 50 años o menores de 20.
\item Obtén todos los resultados cuyo email acabe en \emph{.org}.
\item Igual que antes pero que incluya tanto a los menores o igual de 50 y además su correo acabe en \emph{.org}
\item Añade un nuevo objeto con email correo\@email.es, nombre Pedrito Pérez, edad 21 y rol Webmaster.
\item Actualiza la edad del dato anterior a 21, usando el nombre como patrón de búsqueda.
\item Finalmente borra dicho objeto con el email.
\end{itemize}

\question(3.5 ptos) Se te aporta un servicio web con arquitectura REST. Con \emph{node.js, express, mongodb y mongoose} que va a usar la BD de datos anterior y la colección del ejercicio anterior.
\begin{parts}

\part Modifica el modelo para que tenga en cuenta:
\begin{itemize}
\item La edad debe ser mayor que 18
\item El correo sea único y como valor por defecto email@email.es
\item El rol solo pueda tomar los siguientes valores: \emph{Consultant, CEO, Secretary, Translator, Typist, Webmaster, Engineer, Prueba)}. Por defecto Webmaster.
\item 
\end{itemize}
\part Crea una nueva ruta llamada \emph{examenB} que responda a las sig1uientes funcionalidades:
\begin{itemize}
\item Obtener los objetos dada una edad determinada y su correo acabe en un valor determinado. Si pruebas con 50 y .com y obtendras:
\begin{enumerate}
\item Ransom Bruhke,50,rbruhkei@bigcartel.com,Engineer
\item Tyrus Gerbel,50,tgerbel15@time.com,Engineer
\end{enumerate}
\item Actualizar el correo a correo\@correo.es a aquellos que tenga como rol \emph{CEO}

\end{itemize}
\end{parts}
\end{questions}
\newpage
Entrega del ejercicio: el ejercicio se entregará en un documento comprimido con el formato: apellidosNombre.tar.gz o apellidosNombre.zip con la siguiente estrucutura:
\begin{itemize}
\item Se crearán tres directorios, uno denominado \emph{ejercicio1}, \emph{ejercicio2} y otro \emph{ejercicio3}.
\item En la carpeta ejercicio1 se entregara los dos archivos de \emph{javascript}, el que se te aporta mas el closure que tu hagas
\item En la carpeta ejercicio2 se entregará un fichero de texto plano con las consultas mas el comando de importación del array de json.
\item En el ejercicio3 se entregara el servicio web modificado. Como se aporta el fichero \emph{package.json} no hace falta aportar el directorio \emph{node\_modules}
\end{itemize}
\vspace{0,5cm}
\hrule
\vspace{0,5cm}
Documentación a utilizar:
\begin{itemize}
\item \href{https://www.w3schools.com/js/}{Documentación de \emph{javascript}}
\item \href{http://www.etnassoft.com/2016/12/02/introduccion-a-la-poo-en-javascript-moderno-las-nuevas-clases-en-es6/}{Clases con \emph{javascript}}
\item \href{https://docs.mongodb.com/manual/crud/}{CRUD con \emph{mongodb}}
\item \href{https://mongoosejs.com/docs/schematypes.html}{Validación \emph{mongoose}}
\item \href{https://www.codementor.io/olatundegaruba/nodejs-restful-apis-in-10-minutes-q0sgsfhbd}{Aplicación REST con node, express y mongo}
\end{itemize}
\end{document}
