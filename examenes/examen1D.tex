\documentclass[addpoints,12pt]{exam}
\usepackage[spanish]{babel}
%\usepackage[latin1]{inputenc}
\usepackage[utf8x]{inputenc}
\usepackage{hyperref}
\pagestyle{empty}
\begin{document}
\begin{center}
\fbox{\fbox{\parbox{5.5in}{\centering {\LARGE EXAMEN TIPO D}\\Sigue las intrucciones que marca cada ejercicio, no solo para la realización del mismo sino también para la entrega de los mismos.\\\emph{No se permite ningun documento de ayuda, excepto los aportados por el profesor}\\\textbf{No se corregira ejercicios con errores de programación}}}}
\end{center}
\vspace{0.1in}
%\makebox[\textwidth]{Nombre:\enspace\hrulefill}
\begin{questions}
\question(4 ptos) Crea los siguientes programas en javascript:
\begin{parts}
\part
(2 ptos) Un cierre que pase como argumentos un \emph{array} de objetos usuario con los siguientes campos (iguales que los datos del fichero de \emph{arrays} de \emph{json} que se te proporiciona):
\begin{itemize}
\item name
\item race
\item gender
\item skill
\end{itemize}
Y realice las siguientes operaciones
\begin{itemize}
\item Nos de el número de usuarios dado un \emph{skill} determinado
\item Nos de un array de objetos usuario dado los atributos \emph{race y gender}
\item Nos de un objetos usuario dado el atributo \emph{name}
\end{itemize}
(2 ptos) Un programa denominada \emph{main.js} que lea el fichero \emph{examen1D.json} que consta de un array de \emph{json} sobre datos de personas. Para la lectura solo tienes que usar el método \emph{readFile} con la correspondiente \emph{callback} y parsear los datos solamente con \emph{JSON.parse}. (No hace falta liar tanto código como en el ejemplo realizado en clase con un fichero de tipo \emph{CSV}).\\
Una vez realizada dicha lectura, el código debe hacer:
\begin{itemize}
\item Importar el \emph{cierre} anterior.
\item Llamar a las funciones del interfaz del cierre y mostrar los datos en consola
\end{itemize}
\end{parts}

\newpage

\question(4 ptos) Servicio web con arquitectura REST. Con \emph{node.js, express, mongodb y mongoose}
\begin{parts}

\part Crea una base de datos en \emph{mongodb} denominada examen. Importa los datos del fichero \emph{examen1D.json} a una colección de dicha base de datos, llama a dicha colección \emph{users}
\part Crea una aplicación RESTful con las siguientes acciones, dichas acciones corresponde a rutas del servicio web usando como persistencia la base de datos de \emph{mongodb} anterior.
\begin{itemize}
\item Obtener todos los documentos dado el atributo \emph{gender}
\item Crear un documento nuevo
\item Borrar un documento dado el atributo \emph{name}.
\end{itemize}
\part Se valorara el uso de modularidad y la creación de un fichero \emph{package.json}
\end{parts}
\end{questions}
\vspace{0,5cm}
\hrule
\vspace{0,5cm}
Entrega del ejercicio: el ejercicio se entregará en un documento comprimido con el formato: apellidosNombre.tar.gz o apellidosNombre.zip con la siguiente estrucutura:
\begin{itemize}
\item Se crearán dos directorios, uno denominado \emph{ejercicio1} y otro \emph{ejercicio2}.
\item En el ejercicio1 se entregara los archivos de \emph{javascript}
\item En el ejercicio2 se entregara los archivos correspondientes. En el caso que se aporte el fichero \emph{package.json} no hace falta aportar el directorio \emph{node\_modules}
\end{itemize}
\vspace{0,5cm}
\hrule
\vspace{0,5cm}
Documentación a utilizar:
\begin{itemize}
\item \href{https://www.w3schools.com/js/}{Documentación de \emph{javascript}}
\item \href{https://nodejs.org/dist/latest-v8.x/docs/api/}{Documentación \emph{node.js}}
\item \href{https://docs.mongodb.com/manual/crud/}{CRUD con \emph{mongodb}}
\item \href{http://expressjs.com/es/}{Documentación \emph{express}}
\item \href{http://mongoosejs.com/docs/guide.html}{Documentación \emph{mongoose}}
\item \href{https://docs.npmjs.com/}{Documentación \emph{npm}}

\end{itemize}
\end{document}
