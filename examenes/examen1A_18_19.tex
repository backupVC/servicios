\documentclass[addpoints,12pt]{exam}
\usepackage[spanish]{babel}
%\usepackage[latin1]{inputenc}
\usepackage[utf8x]{inputenc}
\usepackage{hyperref}
\pagestyle{empty}
\begin{document}
\begin{center}
\fbox{\fbox{\parbox{5.5in}{\centering {\LARGE EXAMEN TIPO A}\\Sigue las intrucciones que marca cada ejercicio, no solo para la realización del mismo sino también para la entrega de los mismos.\\\emph{No se permite ningun documento de ayuda, excepto los aportados por el profesor}\\\textbf{No se corregira ejercicios con errores de programación}}}}
\end{center}
\vspace{0.1in}
%\makebox[\textwidth]{Nombre:\enspace\hrulefill}
\begin{questions}
\question(3.5 ptos) Crea un cierre o (\emph{closure}) que pase como argumentos un \emph{array} de objetos persona con los siguientes campos (iguales que los datos del fichero de \emph{arrays} de \emph{datos.csv} que se te proporciona):
\begin{itemize}
\item name
\item age
\item email
\item rol
\end{itemize}
Y realice las siguientes operaciones
\begin{itemize}
\item Nos de la edad media de las personas.
\item Nos de un objeto persona dado el email personal.
\item Nos de un array de objetos persona dado un rol determinado.
\end{itemize}
Se aporta un fichero denominado \emph{datos.csv} y un programa \emph{leerCSV.js} que se encarga de leer los datos y crear un array de objeto persona que debes usar para comprobar los métodos del closure. Este fichero (\emph{leerCSV.js}) no debes modificar la lectura del fichero CSV, el se encarga de eliminar la cabecera, parsear todas las líneas y construye el array de objetos que vas a usar\\
Como datos de comprobación puedes usar:\\
El resultado de la edad media es 42.24\\
Si introduces el email jdelafield2r\@facebook.com te debe aparecer el objeto (Jaine DelaField,41,jdelafield2r\@facebook.com,Engineer)\\
En el último método si introducimos \emph{Engineer} debes obtener un array de tamaño 53, en el caso de \emph{Webmaster} es de 24

\newpage

\question(3 ptos) Se te proporciona un fichero denominado \emph{datos.json} que es un array de json. Importa dichos datos a la BD de mongo, usa como nombre de BD \emph{test} y coleccion \emph{users}.
\begin{itemize}
\item Obtén todos los resultados de objetos que sean mayores de 50 años.
\item Exactamente igual que antes y ademas su email acabe en \emph{.com}.
\item Igual que antes pero que incluya tanto a los mayores de 50 o su correo acabe en \emph{.com}
\item Añade un nuevo objeto con email correo\@correo.es, nombre Pedrito Pérez, edad 51 y rol Engineer.
\item Actualiza la edad del dato anterior a 61, usando el email como patrón de búsqueda.
\item Finalmente borra dicho objeto con el email.
\end{itemize}

\question(3.5 ptos) Se te aporta un servicio web con arquitectura REST. Con \emph{node.js, express, mongodb y mongoose} que va a usar la BD de datos anterior y la colección del ejercicio anterior.
\begin{parts}

\part Modifica el modelo para que tenga en cuenta:
\begin{itemize}
\item La edad quede comprendida entre 18 y 70 años.
\item El rol solo pueda tomar los siguientes valores: \emph{Consultant, CEO, Secretary, Translator, Typist, Webmaster, Engineer, Prueba)}. Por defecto CEO.
\item Todos los valores sor obligatorios.
\end{itemize}
\part Crea una nueva ruta llamada \emph{examenA} que responda a las siguientes funcionalidades:
\begin{itemize}
\item Obtener un objeto dado el email
\item Actualizar los objetos de una edad determinada a rol Prueba. Puedes probar con 50 años y acutalizará estos tres objetos:
\begin{enumerate}
\item Andres Raynes,50,araynesg\@who.int,CEO
\item Ransom Bruhke,50,rbruhkei\@bigcartel.com,Engineer
\item Tyrus Gerbel,50,tgerbel15\@time.com,Engineer

\end{enumerate}
\end{itemize}
\end{parts}
\end{questions}
\newpage
Entrega del ejercicio: el ejercicio se entregará en un documento comprimido con el formato: apellidosNombre.tar.gz o apellidosNombre.zip con la siguiente estrucutura:
\begin{itemize}
\item Se crearán tres directorios, uno denominado \emph{ejercicio1}, \emph{ejercicio2} y otro \emph{ejercicio3}.
\item En la carpeta ejercicio1 se entregara los dos archivos de \emph{javascript}, el que se te aporta mas el closure que tu hagas
\item En la carpeta ejercicio2 se entregará un fichero de texto plano con las consultas mas el comando de importación del array de json.
\item En el ejercicio3 se entregara el servicio web modificado. Como se aporta el fichero \emph{package.json} no hace falta aportar el directorio \emph{node\_modules}
\end{itemize}
\vspace{0,5cm}
\hrule
\vspace{0,5cm}
Documentación a utilizar:
\begin{itemize}
\item \href{https://www.w3schools.com/js/}{Documentación de \emph{javascript}}
\item \href{https://docs.mongodb.com/manual/crud/}{CRUD con \emph{mongodb}}
\item \href{https://mongoosejs.com/docs/schematypes.html}{Validación \emph{mongoose}}
\item \href{https://www.codementor.io/olatundegaruba/nodejs-restful-apis-in-10-minutes-q0sgsfhbd}{Aplicación REST con node, express y mongo}
\end{itemize}
\end{document}
