\documentclass[4paper]{article}
\usepackage[spanish]{babel}
%\usepackage[ansinew]{inputenc}
\usepackage[utf8x]{inputenc}
%\usepackage[utf-8]{inputenc}
%\usepackage[T1]{fontenc}
\usepackage{graphicx}
\usepackage{multicol}
\usepackage{float}
\usepackage{hyperref} 
%\usepackage{longtable}
%\usepackage{array}
%\usepackage{multirow}
%\usepackage[latin1]{inputenc}
%\inputencoding{latin}
\newcommand{\J}{Java}
\newcommand{\p}{procesos}
\newcommand{\s}{procesos}

%\newcommand{\j}{JavaScript }
%%%%%%%%%%%%%%%%%%%%%%%%%%%%%%%%%%%%%
%Para escribir codigo en latex
\usepackage{listings}
\usepackage{color}


\definecolor{lightgray}{rgb}{.9,.9,.9}
\definecolor{darkgray}{rgb}{.4,.4,.4}
\definecolor{purple}{rgb}{0.65, 0.12, 0.82}

\lstdefinelanguage{JavaScript}{
  keywords={typeof, new, true, false, catch, function, return, null, catch, switch, var, if, in, while, do, else, case, break},
  keywordstyle=\color{blue}\bfseries,
  ndkeywords={class, export, boolean, throw, implements, import, this},
  ndkeywordstyle=\color{darkgray}\bfseries,
  identifierstyle=\color{black},
  sensitive=false,
  comment=[l]{//},
  morecomment=[s]{/*}{*/},
  commentstyle=\color{purple}\ttfamily,
  stringstyle=\color{red}\ttfamily,
  morestring=[b]',
  morestring=[b]"
}

\lstset{
   language=JavaScript,
   backgroundcolor=\color{lightgray},
   extendedchars=true,
   basicstyle=\footnotesize\ttfamily,
   showstringspaces=false,
   showspaces=false,
  % numbers=left,
   numberstyle=\footnotesize,
   numbersep=9pt,
   tabsize=2,
   breaklines=true,
   showtabs=false,
   captionpos=b
}
%%%%%%%%%%%%%%%%%%%%%%%%%%%%%%%%%%%%%%%%
\renewcommand{\tablename}{Tabla}
\renewcommand{\S}{ \p}
\author{Manuel Molino Milla}
\title{\textbf{Mongoose}}
\date{\today}

\begin{document}
\maketitle 
\tableofcontents
\newpage

\section{Introducción}

\begin{itemize}
\item \emph{Moongose} es un modulo de \emph{node.js} 
\item Hace de puente entre \emph{node.js} y \emph{mongodb}
\item Trabaja como un \emph{ORM} (Object-Relational mapping), que es una técnica de programación para convertir datos entre el sistema de tipos utilizado en un lenguaje de programación orientado a objetos y la utilización de una base de datos.
\end{itemize}

\subsection{Instalacion}
\begin{lstlisting}
npm install mongoose
\end{lstlisting}

\subsection{Uso en node.js}
\begin{lstlisting}
var mongoose = require('mongoose');
\end{lstlisting}

\subsection{Conexión a la BD}
\begin{lstlisting}
mongoose.mongoose.connect('mongodb://localhost/myappdatabase');
\end{lstlisting}


\subsection{Esquemas}
Todo en \emph{mongoose} comienza con un esquema. Cada esquema \emph{mapea} la coleccion definida en \emph{MongoDB} y define la forma de acceder a sus datos. Ejemplo: 
\begin{lstlisting}
var mongoose = require('mongoose');
var Schema = mongoose.Schema;

var blogSchema = new Schema({
  title:  String,
  author: String,
  body:   String,
  comments: [{ body: String, date: Date }],
  date: { type: Date, default: Date.now },
  hidden: Boolean,
  meta: {
    votes: Number,
    favs:  Number
  }
});
var Blog =  mongoose.model('Blog', blogSchema);
//importante Blog mayuscula de blog que
//es la coleccion en la base de datos-
\end{lstlisting}
\newpage
Los tipos permitidos en este esquema son:
\begin{itemize}
\item String
\item Number
\item Date
\item Buffer
\item Boolean
\item Mixed
\item Objectid
\item Array
\end{itemize}
\newpage
Ejemplo:
\begin{lstlisting}
var schema = new Schema({
  name:    String,
  binary:  Buffer,
  living:  Boolean,
  updated: { type: Date, default: Date.now },
  age:     { type: Number, min: 18, max: 65 },
  mixed:   Schema.Types.Mixed,
  _someId: Schema.Types.ObjectId,
  array:      [],
  ofString:   [String],
  ofNumber:   [Number],
  ofDates:    [Date],
  ofBuffer:   [Buffer],
  ofBoolean:  [Boolean],
  ofMixed:    [Schema.Types.Mixed],
  ofObjectId: [Schema.Types.ObjectId],
  ofArrays:   [[]]
  ofArrayOfNumbers: [[Number]]
  nested: {
    stuff: { type: String, lowercase: true, trim: true }
  }
})

// example use

var Thing = mongoose.model('Thing', schema);

var m = new Thing;
m.name = 'Statue of Liberty';
m.age = 125;
m.updated = new Date;
m.binary = new Buffer(0);
m.living = false;
m.mixed = { any: { thing: 'i want' } };
m.markModified('mixed');
m._someId = new mongoose.Types.ObjectId;
m.array.push(1);
m.ofString.push("strings!");
m.ofNumber.unshift(1,2,3,4);
m.ofDates.addToSet(new Date);
m.ofBuffer.pop();
m.ofMixed = [1, [], 'three', { four: 5 }];
m.nested.stuff = 'good';
m.save(callback);
\end{lstlisting}
\newpage
\begin{lstlisting}
var schema1 = new Schema({
  test: String // `test` is a path of type String
});

var schema2 = new Schema({
  test: { type: String } // `test` is a path of type string
});
\end{lstlisting}


\begin{lstlisting}
var schema2 = new Schema({
  test: {
    type: String,
    lowercase: true // Always convert `test` to lowercase
  }
});
\end{lstlisting}

\begin{lstlisting}
var numberSchema = new Schema({
  integerOnly: {
    type: Number,
    get: v => Math.round(v),
    set: v => Math.round(v),
    alias: 'i'
  }
});

var Number = mongoose.model('Number', numberSchema);

var doc = new Number();
doc.integerOnly = 2.001;
doc.integerOnly; // 2
doc.i; // 2
doc.i = 3.001;
doc.integerOnly; // 3
doc.i; // 3
\end{lstlisting}

\begin{lstlisting}
var schema2 = new Schema({
  test: {
    type: String,
    index: true,
    unique: true // Unique index. If you specify `unique: true`
    // specifying `index: true` is optional if you do `unique: true`
  }
});
\end{lstlisting}
\newpage
\subsection{Creando datos}
\begin{lstlisting}
var User = require('./user');

var newUser = User({
  id : 2001,
  sexo : 'Male',
  edad : 100,
  nombre : "Nuevo usuario desde  moongose"
});

newUser.save(function(err) {
  if (err) throw err;
  console.log("Usuario creado");
});
\end{lstlisting}


\subsection{Leyendo datos}
Obteniendo todos los datos:
\begin{lstlisting}
// get all the users
User.find({}, function(err, users) {
  if (err) throw err;

  // object of all the users
  console.log(users);
});
\end{lstlisting}
Obteniendo un dato determinado:
\begin{lstlisting}
// get the user starlord55
User.find({ username: 'starlord55' }, function(err, user) {
  if (err) throw err;

  // object of the user
  console.log(user);
});
\end{lstlisting}
Buscando por \emph{id}
\begin{lstlisting}
// get a user with ID of 1
User.findById(1, function(err, user) {
  if (err) throw err;

  // show the one user
  console.log(user);
});
\end{lstlisting}
Consultas:
\begin{lstlisting}
User.find({first_name : /^M/, id : {$gt : 800}}, function (err, data){
  if (err) throw err;
  console.log(data);
});
\end{lstlisting}

\subsection{Actualizaciones}
Buscar y actualizar un registros:
\begin{lstlisting}
// find the user starlord55
// update him to starlord 88
User.findOneAndUpdate({ username: 'starlord55' }, { username: 'starlord88' }, function(err, user) {
  if (err) throw err;

  // // muestra el antiguo
  console.log(user);
  // con {new: true}
  // muestra el nuevo
});
\end{lstlisting}

Buscar y actualizar múltiples registros:
\begin{lstlisting}
User.update({first_name : 'Manuel'}, {last_name : 'Multiples cambios'}, 
       {multi : true}, function (err, datos) {
                if (err) throw err;
                console.log(datos);
});
\end{lstlisting}

Buscar por \emph{id} y actualizar
\begin{lstlisting}
// find the user with id 4
// update username to starlord 88
User.findByIdAndUpdate(4, { username: 'starlord88' }, function(err, user) {
  if (err) throw err;

  console.log(user);
  
});
\end{lstlisting}


\subsection{Borrado de datos}
Buscar y eliminar a la vez:
\begin{lstlisting}
User.findOneAndRemove({ username: 'starlord55' }, function(err) {
  if (err) throw err;

  // we have deleted the user
  console.log('User deleted!');
});
\end{lstlisting}

Buscar por \emph{id} y borrar:
\begin{lstlisting}
// find the user with id 4
User.findByIdAndRemove(4, function(err) {
  if (err) throw err;

  // we have deleted the user
  console.log('User deleted!');
});
\end{lstlisting}

\subsection{Mas ejemplos}
\href{http://mongoosejs.com/docs/queries.html}{consultas}

\section{Ejercicios}
\begin{itemize}
\item En la colección usada en el tema de \emph{MongoDB} inserta un nuevo dato con valor de \emph{id} igual a \emph{2000}
\item Realizando una búsqueda en la colección, indica cuantas personas tienen 100 años de edad.
\item Indica las personas que su edad está comprendida entre 100 y 98.
\item Busca aquellos documentos que corresponda a personas que sean menores de edad y su nonbre empieza por vocal.
\item Actualiza el documento que le corresponde un \emph{id} igual a 2000 y cambia su edad a 18.
\item Finalmente borra este documento.
\end{itemize}
\end{document}
